\section{Parallel architecture design}

The parallel version consists in a pipeline of three steps:
\begin{enumerate}
    \item pick one by one the windows from the input stream and pushes them in an input queue;
    \item pops the tasks from the input queue, compute the skyline of the windows and pushes the skylines in an output queue;
    \item pops and consumes skylines from the output queue.
\end{enumerate}

\noindent
Since the second step is the one that can scale, it can also be seen as an emitter that picks the windows, a set of workers that compute the skylines and a collector that consumes the skylines, i.e. a pipeline with a single node, a self-organized farm, and another single node.

\bigskip\noindent
\begin{figure}[h]
    \centering
    \input{assets/parallel_diagram}
    \bigskip
    \caption{Parallel architecture diagram}
    \label{fig:parallel_diagram}
\end{figure}

\noindent
We considered also other patterns, but they were discarded for poor performances.

For example, we can simplify the architecture explained above by reducing it to a farm where workers pick windows and consume skylines by theirself, but extracting a windows is a heavier task than popping it from a list, and is not a good idea to do that while holding a lock.

Another option was to switch from a self-organized farm to an emitter-collector farm, where the emitter pushes the windows to a worker-dedicated queue instead of the shared queue. The same for the collector. Anyway, it turns out that the real bottleneck was the memory and this implementation increase the overhead. Since the performance doesn't increase even pushing more windows at time to the workers, also this version was discarded.